% !TEX root = ./main.tex

\section{Commercieel plan} % (fold)
\label{sec:commercieel_plan}

\subsection{Mirror} % (fold)
\label{sub:mirror}
Mirror is een innovatieve dienstverlener in de schaduw van ontwerpers en makers. Heb je als maker een idee in je hoofd of al een gedetailleerd ontwerp? Mirror ondersteunt je in elke fase van het ontwerp tot de uitvoering bij het realiseren van 3D-projecten via 3D-printen, lasercutting, …

Mirror kiest voor een sterke verbondenheid met de regio Brugge en werkt aan een duurzame klantenbinding door co-creatie. Mirror laat makers schitteren.
% subsection mirror (end)

\subsection{Mirror in Brugge} % (fold)
\label{sub:mirror_in_brugge}
Mirror kiest Brugge als vestigingsplek. Brugge richt de blik op morgen om zo het stoffige erfgoedimago af te schudden en niet geassocieerd te worden met het louter opvoeren van ambachten uit het verleden ter vermaak van een toeristisch publiek. Toekomst maken voor Brugge betekent immers ruimte geven aan experiment, zonder het verleden te vergeten. Een rijke traditie als gedroomde voedingsbodem voor innovatie! De Stad Brugge wil handwerk in de brede zin van het woord voor het voetlicht brengen en weer opwaarderen.\cite{blueprint-handmade}

Brugge heeft niet alleen een bijzonder rijke traditie in verfijnd handwerk maar is vandaag de thuisbasis voor vele eigentijdse vakmensen, ontwerpers en ambachtelijke ondernemingen (kalligrafie en letterbeeldhouwen, gastronomie, kant, restauratie en hedendaags ambachtelijk design). Mirror is in Brugge een ideale partner voor de reeds aanwezige en er zich in de toekomst vestigende makers.
% subsection mirror_in_brugge (end)

\subsection{Mirror als meerwaarde in een groeiende markt} % (fold)
\label{sub:mirror_als_meerwaarde_in_een_groeiende_markt}
Brugge kiest ervoor ambachten kwalitatief in de kijker te zetten. Mirror kan de traditionele ambachten van de makers naar een hoger plan brengen door er eigentijdse innovatieve accenten aan toe te voegen. Hierdoor worden de reeds aanwezige ambachtelijke ondernemingen hipper en spreken hierdoor een specifiek marktsegment aan. De groeikansen zijn een eigentijdser, hipper en jonger city-hoppend publiek aan te trekken. Brugge trekt deze kaart en wordt sterker, profileert zich. Brugge meet zich zo een hedendaags imago aan  als levende cultuurstad van makers. Een stad die vanuit een rijk verleden knipoogt naar de toekomst en kan zich zo onderscheiden van andere Vlaamse centrumsteden zoals Kortrijk (design) en Antwerpen (mode). Mirror kan niet enkel mee genieten van deze nieuwe impulsen, maar versterkt deze door het innovatieve karakter.
% subsection mirror_als_meerwaarde_in_een_groeiende_markt (end)

\subsection{Fase in productlevenscyclus} % (fold)
\label{sub:fase_in_productlevenscyclus}
Mirror situeert zich in een nieuw marktsegment. Het combineert een dienst (van idee, ontwerp naar uitvoering) met een product (3D-print, lasercut\dots).

\subsubsection{Dienst: van idee of ruw ontwerp naar uitvoering} % (fold)
\label{ssub:dienst_van_idee_of_ruw_ontwerp_naar_uitvoering}
Deze dienst creëert een behoefte bij de potentiële klant (de makers). Het situeert zich in de levensfase van het lanceren. Mirror besteedt voldoende aandacht aan het zichzelf in de markt plaatsen in nauwe samenwerking met de stad Brugge enerzijds en de potentiële klanten anderzijds.
% subsubsection dienst_van_idee_of_ruw_ontwerp_naar_uitvoering (end)

\subsubsection{Product: 3D-printing, lasercutting} % (fold)
\label{ssub:product_3d_printing_lasercutting}
In de productiecyclus situeert het product situeert zich meer in de groei van het product. 3D-oplossingen zijn relatief nieuw, maar bij professionals niet onbekend. Binnen de co-creatie met de makers besteedt Mirror voldoende aandacht aan de wisselwerking tussen ontwerper en uitvoerder om zo tot een win-win situatie en kwaliteitsverhoging bij beide partijen te komen.
% subsubsection product_3d_printing_lasercutting (end)
% subsection fase_in_productlevenscyclus (end)

\subsection{Technische aspecten: gebruikte technologieën} % (fold)
\label{sub:technische_aspecten_gebruikte_technologieën}

\subsubsection{Dienst: van idee of ruw ontwerp naar uitvoering} % (fold)
\label{ssub:dienst_van_idee_of_ruw_ontwerp_naar_uitvoering_technisch}

Vanuit een sterkte-zwakteanalyse tussen bestaande programma’s kiest Mirror voor het gebruiken van zowel VectorWorks (2017) als AutoCAD (2017). Dit is omdat er verschillende use-cases zijn voor ontwerp in 3D als in uitwerken van 2.5D ontwerp met een lasercutter.

Hoewel dit een relatief grote kost is (meer dan 4000€ per jaar), is het zeer handig om te kunnen kiezen voor de juiste toepassing voor verschillende doeleinden.

Ook is het nodig om een goeie computer aan te schaffen om snel te kunnen werken.

% subsubsection dienst_van_idee_of_ruw_ontwerp_naar_uitvoering_technisch (end)

\subsubsection{Productieproces: make or buy} % (fold)
\label{ssub:productieproces_make_or_buy}
Mirror kiest ervoor om in de eerste fase van het ondernemen de regionale markt te verkennen. Het unieke van Mirror is de dienstverlening. Om de investeringen haalbaar te houden en op de markt af te stemmen onderzocht Mirror wanneer te investeren in 3D en laserprinters of we eerder kiezen in samenwerkingsverbanden met reeds bestaande 3D en laserprintcentra in en buiten de regio. We gaan hierbij ook op zoek naar alternatieve samenwerkingsverbanden (bv. Fab-Labs, particulieren…).

Daar de innovatie van Mirror in het uitwerken van het idee, ontwerp van de maker is, wordt pas in een latere fase van de onderneming een gericht investeringsplan opgemaakt op basis van de noden van klanten en hoe we hier gericht op kunnen inspelen.

% subsubsection productieproces_make_or_buy (end)
% subsection technische_aspecten_gebruikte_technologieën (end)

\subsection{Vergelijking van bestaande concurrenten} % (fold)
\label{sub:vergelijing_van_bestaande_concurrenten}

Concurrenten in de Brugse regio zijn 3D-print- en lasercuttingcentra die ook een ontwerpdienst aanbieden. Het grote verschil is echter dat zijn hun focus leggen op het exact doorgeven van wat de klant vraagt, zonder daarin in co-creatie te gaan.

Volgens Mirror is de samenwerking met de makers net de ``missing link'' tussen het enkel uitvoerende van een al gemaakt ontwerp en ambachten. Omdat ambachtslui wel de ideeën hebben, maar niet noodzakelijk de technische vaardigheid hebben om dingen die niet met traditionele materialen gemaakt kunnen worden volledig uit te werken.

% subsection vergelijing_van_bestaande_concurrenten (end)
\subsection{Marktonderzoek} % (fold)
\label{sub:marktonderzoek}

\subsubsection{SWOT-analyse} % (fold)
\label{ssub:swot_analyse}

\begin{table}[H]
\caption{SWOT-analyse}\label{table:swot}
\begin{tabulary}{\textwidth}{C|C|C}
& \textbf{Strengths} & \textbf{Weaknesses} \\\hline
\multirow{7}{*}{\rotatebox{90}{\textbf{Internal Factors}}} & kennis van 3D-programma’s & weinig basiskapitaal \\
& kennis van Brugse regio & hoge prijs nodig om kosten te dekken \\
& kan mij ten dienste stellen van anderen, wil zelf niet uitblinken & weinig basiskapitaal \\
& uniek concept (dienst koppelen aan productie) & eenmanszaak in opstart \\
& kwaliteit & \\
& passie & \\
& netwerk & \\\hline
\multirow{4}{*}{\rotatebox{90}{\textbf{External Factors}}} & samenwerking met partners zoals `Handmade in Brugge', stad Brugge … & kwaliteit wordt niet steeds door klanten als ijkpunt genomen \\
& groeimarkt (Brugge als cultuurstad) & popularisering 3D-oplossingen \\
& toerisme als groeisector & economische stagnatie \\
& & nichemarkt (klein) \\\hline
& \textbf{Opportunities} & \textbf{Threaths} \\
\end{tabulary}
\end{table}

% subsubsection swot_analyse (end)

% subsection marktonderzoek (end)

\subsubsection{Prijszetting} % (fold)
\label{ssub:prijszetting}
Mirror legt het accent op kwaliteit en een duurzame relatie met makers. Niettegenstaande de investeringskost in de eerste fase van de onderneming beperkt is, houden we rekening  met de nodige investeringen die in een latere fase nodig zijn. De concrete prijssetting wordt verder uitgewerkt in de volgende hoofdstukken.  Juiste hoofdstuk en titel vermelde
Daar we als onderneming de ontwikkeling van ons concept simultaan laten verlopen met  de ontwikkeling ervan is de prijszetting een uitdaging. We volgen de principes van lean ondernemen (zie hoofdstuk \ref{sec:actieplan}).

\begin{table}[H]
\caption{Waardeniveaus}\label{table:waardeniveaus}
\begin{tabulary}{\textwidth}{CCC}
\textbf{waarde} & \textbf{toelichting} & \textbf{mirror} \\ \hline
Features & Wat is de minimumwaarde van de dienst/product? & 3D-oplossing (uitwerken, uitvoeren) \\ \hline
Advantages & Welke unieke voordelen bied je? Hoe ‘anders’ is jouw aanbod ten opzichte van de concurrentie? & Combinatie uitwerken, uitvoeren, verschillende 3D-oplossingen binnen één onderneming \\ \hline
Benefits & Welke impact veroorzaak je bij de klant? Hoe ‘wint’ je klant bij jouw zaak? & Meerwaarde eigen product \\ \hline
Benefit of the benefits & Hoe verbeter je het leven van je klant? & Eigentijds accent product, droomproducten technisch realiseren
\end{tabulary}
\end{table}

Mirror onderzocht onderstaande mogelijke strategieën\cite{lean-boekje} om de prijszetting te bepalen. Zie ook bijlage \ref{sub:prijsstrategieën} voor een gedetailleerde uitleg over de verschillende strategieën.

\paragraph{Klantengesprekken}

Daar we voor een nieuwe en kleine markt kiezen, is deze pricing strategie minder voor de hand liggend. We bevragen klanten naar onze vooropgezette prijs. Wanneer er een discrepantie is tussen verwachtte prijs en dienst/product, passen wij de waarde aan (zie tabel \ref{table:waardeniveaus}: waardeniveaus).

\paragraph{Break-even draaien}

Mogelijke omzet is moeilijk in te schatten. We proberen uiteraard rekening te houden met mogelijke kosten

\paragraph{Buy vs. build}

Lange termijn prijszetting is niet haalbaar. Hangt af van de intensiteit van samenwerken. Biedt mogelijkheden naar de toekomst (bv. voor vast bedrag per maand krijgt de maker een samen ontwikkeld en uitgevoerde  3D-oplossing per 3 maanden)

\paragraph{Prijzenspectrum}

Beperkte vergelijking mogelijk daar dienst relatief nieuw is. We plaatsen de bestaande concurrenten die producten of gelijkaardige diensten verlenen op een prijzenlijn (zie \ref{sub:grondige_studie_van_bestaande_en_potentiële_concurrenten} Potentiële concurenten)


\paragraph{Multiaxis pricing (prijs en waardes op klanten afstemmen)}

Zeer bruikbaar gezien de verscheidenheid binnen doelgroep. We nemen multiaxis pricing mee als strategie in onze onderneming

\paragraph{Korting, pilootklanten, trialversie, freemium}

Gratis trial en freemium zijn voor Mirror geen optie, gezien we een fysieke dienst aanbieden.

Onze eerste targetgroep zien we als onze pilootklanten en  we werken met gratis teasers om onze mogelijkheden te tonen en om ons in de markt te zetten.

Daar we inzetten op kwaliteit leveren voor een eerlijke, maar geen dumpingprijs, nemen we de principes van kortingen niet mee. We zetten bewust in op multiaxis pricing (zie eerder)

\paragraph{Besluit} % (fold)
\label{par:besluit}
Mirror zet bij prijszetting in op multiaxis pricing gezien de verscheidenheid binnen de doelgroep. Onze eerste targetgroep (erkende makers `Handmade in Brugge') zien we als onze pilootklanten en we werken met gratis teasers om zo de gekozen targetgroep aan te trekken, onze mogelijkheden te tonen en om ons in de markt te zetten. Mirror staat voor kwaliteit en samenwerking. We bepalen een eerlijke prijs voor onze dienst/product zonder te vervallen in onhaalbare dumpingprijzen. Onze concurrenten brengen we in kaart i.f.v. samenwerking (bv. 3D-printcentra worden onze leveranciers), we schatten hun geboden diensten in en houden rekening met hun prijszetting bij het bepalen van onze prijs. Bij het bepalen van onze prijs houden we uiteraard rekening met onze investeringskost (hardware, software), onze opstartkosten (pop-up, productie teasers, organisatie evenementen, inrichtingskosten, personeelskost…) en vaste kosten (kantoren, internet, elektriciteit/gas/water, meubilair…). We sturen onze prijsstrategie indien nodig bij.
% paragraph besluit (end)

% subsubsection prijszetting (end)

\subsection{Organisatievorm} % (fold)
\label{sub:organisatievorm}
In de opstartfase kiest Mirror ervoor om een eenmanszaak te zijn. In de fase van de pop-up, waar we werken aan naambekendheid en ons in de markt plaatsen, versterkt een jobstudent de werking. Afhankelijk van de groei, het succes en de mogelijke uitbreiding naar andere regio’s kan het personeel groeien. Ook hier kiest Mirror voor een grote zelfstandigheid en eigenaarschap van de toekomstige werknemers.
% subsection organisatievorm (end)

\subsection{Studie van potentiële klanten} % (fold)
\label{sub:studie_van_potentiële_klanten}
Mirror onderzocht de doelmarkt van de makers in Brugge en omstreken. Hierbij maakten we gebruik van onze sleutelpartner `Handmade in Brugge'. Deze gegevens vulden we aan met informatie uit gespecialiseerde websites.

\subsubsection{Targetgroep 1: door `Handmade in Brugge' erkende makers} % (fold)
\label{ssub:targetgroep-1}

Zie ook bijlage \ref{sub:door_handmade_in_brugge_erkende_makers} voor een overzicht van alle erkende makers.

De door `Handmade in Brugge' erkende makers zijn terug te brengen naar volgende sectoren:

\begin{itemize}
  \item Textiel/kant
  \item Accessoires
  \item Interieur
  \item Letters en papier
  \item Eten en drinken
  \item Chocolade
  \item Zoet
  \item Andere (muziekinstrumenten, fietsen)
\end{itemize}

Mirror richt zich op het leveren van diensten aan de makers. Voor de eerste 4 categorieën zijn mogelijke samenwerkingen meer voor de hand liggend dan de hierna volgende die zich eerder in culinaire sfeer bevinden. Toch loont het de moeite ook met deze makers in gesprek te gaan om mogelijke samenwerkingen in co-creatie vorm te geven.


\begin{longtabu} to \textwidth {XXX}
\caption{Samenwerking per sector}\label{table:samenwerking-per-sector}\\
\textbf{sectoren} & \textbf{mogelijke samenwerking Mirror} &  \textbf{voorbeelden}\\\hline
Textiel/kant (8) & \multirow{5}{0.3\textwidth}{Door maker uitgedacht accessoire uitwerken in 3D of lasercut} & zelf ontworpen knoop of ornement voor op kledingstuk, hoed, … \\
Accessoires (7) & & in 3D-print uitgewerkt handvat rugzak, door maker ontworpen hak voor schoen, door juwelenontwerper uitgewerkte sluiting …\\
Interieur (12) & & Door meubelmaker ontwerpen zetelpoot, met lasercutting bewerkt glas… \\
Andere (muziekinstrumenten en fietsen) (3) & & Persoonlijk accessoire instrument (plectrum, stemschroef…) \\
Letters en papier (9) & & kalligrafie uitgewerkt in 3D \\\hline
Eten en drinken (6) & \multirow{3}{0.3\textwidth}{accessoire voor het creëren van producten} &  \multirow{3}{0.3\textwidth}{mal voor pralines, taartjes …}\\
Chocolade (6) &  &  \\
Zoet (5) &  &
\end{longtabu}


% subsubsection targetgroep-1 (end)
\subsubsection{Targetgroep 2: niet-erkende makers door `Handmade in Brugge' in zelfde sectoren} % (fold)
\label{ssub:targetgroep-2}
In de Brugse regio zijn meer makers actief in de door `Handmade in Brugge' erkende sectoren. Deze makers zijn de tweede targetgroep van Mirror.
% subsubsection targetgroep-2 (end)
\subsubsection{Targetgroep 3: makers in de Brugse regio in andere sectoren} % (fold)
\label{ssub:targetgroep-3}
Naast de door `Handmade in Brugge' vastgelegde sectoren ziet target nog enkele andere potentiële doelgroepen met wie samenwerkingsverbanden mogelijk zijn. De sectoren waar we mogelijkheden zien zijn de volgende:

\begin{itemize}
  \item architecten
  \item bloemisten
  \item kleermakers
  \item schoenontwerpers
\end{itemize}

Zie ook bijlage \ref{sub:makers_in_de_brugse_regio_in_andere_sectoren} voor een volledig overzicht.

De architecten zijn hierbij een specifieke doelgroep. Hier zal het omzetten van het ontwerp in een CAD programma minder van belang zijn, maar eerder de uitvoering ervan in een 3D-oplossing. Voor de andere zal het tijdens de pop-upperiode mogelijk zijn om kennis te maken en Mirror in te boeken, maar geen specifieke pilootontwerpen worden gemaakt.

% subsubsection targetgroep-3 (end)
\subsubsection{Targetgroep 4: onderwijsinstellingen, studenten en hobbyisten} % (fold)
\label{ssub:targetgroep-4}
Deze specifieke targetgroep bereiken we via de scholen en opleidingscentra. Mirror zet mee in in de toekomst en bouwt mee aan toekomstige makers (professioneel of als vrijetijdsbesteding). In prijszetting kiezen we hier voor andere tarieven dan onze commerciële makers (Zie \ref{sub:prijsstrategieën} Prijsstrategieën, multiaxis pricing). Hierdoor ontdekken toekomstige makers onze mogelijkheden en binden wij hen aan Mirror als toekomstige klant.

Mirror ziet kansen in het samenwerken met scholen en onderwijsinstellingen die toekomstige makers vormen. In samenspraak met docenten en lesgevers kunnen makers in opleiding de kansen leren ontdekken van 3D-oplossingen.

Overzicht onderwijsinstellingen en opleidingscentra: zie bijlage \ref{sub:onderwijsinstellingen_studenten_en_hobbyisten}

% subsubsection targetgroep-4 (end)
% subsection studie_van_potentiële_klanten (end)

% section commercieel_plan (end)
